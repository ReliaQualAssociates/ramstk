\documentclass[letter,12pt]{article}
\usepackage{amssymb} % needed for math
\usepackage{amsmath} % needed for math
\usepackage[utf8]{inputenc}
\usepackage[T1]{fontenc}
\usepackage[margin=2.5cm]{geometry} %layout

\usepackage{hyperref}  % this is needed for forms and links within the text

\hypersetup{
  pdfauthor   = {Doyle Rowland},
  pdfkeywords = {Doyle Rowland, RAMSTK, GUI, test, form},
  pdftitle    = {RAMSTK GUI Test Procedure}
}

\begin{document}

\title{RAMSTK GUI Test Procedure}
\author{Doyle Rowland}
\date{\today}

\section{Summary}
This procedure provides guidance for testing the GUI components of the RAMSTK application. \\

\noindent An edit persists when the widget being edited looses focus and the new text, selection, etc. remains.

\begin{Form}[action=mailto:info@reliaqual.com,encoding=html,method=post]
\section{Revision Module}

\subsection{Module View:}
\CheckBox[name=revisionload,charsize=12pt]{Revision(s) in test database load into module view.} \\

\subsubsection{Edit Actions:}
\CheckBox[name=revisionnoedit,charsize=12pt]{None of the fields in the module view are editable.} \\

\subsubsection{Insert Actions:}
\CheckBox[name=revisioninsert,charsize=12pt]{A new revision can be inserted using the 'Insert Sibling' button in the module view buttonbox.} \\

\subsubsection{Delete Actions:}
\CheckBox[name=revisiondelete,charsize=12pt]{The selected revision can be deleted; user will be prompted to confirm.} \\

\subsubsection{Save Actions:}
\CheckBox[name=revisionsave,charsize=12pt]{The selected revision can be saved with the 'Save' button in the module view buttonbox.} \\

\subsection{Work View:}
\subsubsection{Edit Actions:}
\CheckBox[name=revisioncode,charsize=12pt]{Revision code can be edited and edits persist.} \\
\CheckBox[name=revisionname,charsize=12pt]{Revision name can be edited and edits persist.} \\
\CheckBox[name=revisionremarks,charsize=12pt]{Revision remarks can be edited and edits persist.}

\subsection{List View - Usage Profile:}
\CheckBox[name=usageload,charsize=12pt]{Usage profile in test database loads into list view.}

\subsubsection{Edit Actions:}
\CheckBox[name=missiondescriptionedit,charsize=12pt]{Mission description can be edited and edits persist.} \\
\CheckBox[name=missionunitedit,charsize=12pt]{Mission units can be edited and edits persist.} \\
\CheckBox[name=missionstartedit,charsize=12pt]{Mission start time can be edited and edits persist.} \\
\CheckBox[name=missionendedit,charsize=12pt]{Mission end time can be edited and edits persist.} \\
\CheckBox[name=missionphasedescedit,charsize=12pt]{Mission phase description can be edited and edits persist.} \\
\CheckBox[name=missionphasestartedit,charsize=12pt]{Mission phase start time can be edited and edits persist.} \\
\CheckBox[name=missionphaseendedit,charsize=12pt]{Mission phase end time can be edited and edits persist.} \\
\CheckBox[name=envconditionedit,charsize=12pt]{Environment condition can be edited and edits persist.} \\
\CheckBox[name=envunitedit,charsize=12pt]{Environment units can be edited and edits persist.} \\
\CheckBox[name=envminimumedit,charsize=12pt]{Environment minimum value can be edited and edits persist.} \\
\CheckBox[name=envmaximumedits,charsize=12pt]{Environment maximum value can be edited and edits persist.} \\
\CheckBox[name=envmeanedit,charsize=12pt]{Environment mean value can be edited and edits persist.} \\
\CheckBox[name=envvarianceedit,charsize=12pt]{Environment variance can be edited and edits persist.}

\subsubsection{Insert Actions:}
\CheckBox[name=missionaddsibling,charsize=12pt]{Sibling mission can be added using the 'Insert Sibling' button in the list view buttonbox with another mission selected.} \\
\CheckBox[name=missiophasenaddchild,charsize=12pt]{Child mission phase can be added using the 'Insert Child' button in the list view buttonbox with a missino selected.} \\
\CheckBox[name=missionphaseaddsibling,charsize=12pt]{Sibling mission phase can be added using the 'Insert Sibling' button in the list view buttonbox with a mission phase selected.} \\
\CheckBox[name=envaddchild,charsize=12pt]{Child environment can be added using the 'Insert Child' button in the list view buttonbox with a mission phase selected.} \\
\CheckBox[name=envaddsibling,charsize=12pt]{Sibling environment can be added using the 'Insert Sibling' button in the list view buttonbox with an environment selected.} \\
\CheckBox[name=envnoaddchild,charsize=12pt]{A message dialog is launched when attempting to add a child item to a selected environment notifying the user that environments can't have children.}

\subsubsection{Delete Actions:}
\CheckBox[name=envdelete,charsize=12pt]{The selected environment can be deleted using the 'Remove' button in the list view buttonbox.} \\
\CheckBox[name=missionphasedelete,charsize=12pt]{The selected mission phase and associated environment(s) can be deleted using the 'Remove' button in the list view buttonbox.} \\
\CheckBox[name=missiondelete,charsize=12pt]{The selected mission and associated mission phase(s) can be deleted using the 'Remove' button in the list view buttonbox.}

\subsubsection{Save Actions:}
\CheckBox[name=missionsave,charsize=12pt]{Edits to the selected mission can be saved to the database using the 'Save' button in the list view buttonbox.} \\
\CheckBox[name=missionphasesave,charsize=12pt]{Edits to the selected mission phase can be saved to the database using the 'Save' button in the list view buttonbox.} \\
\CheckBox[name=envsave,charsize=12pt]{Edits to the selected environment can be saved to the database using the 'Save' button in the list view buttonbox.} \\
\CheckBox[name=profilesave,charsize=12pt]{Edits to all elements can be saved to the database using the 'Save All' button in the list view buttonbox.}

\subsection{List View - Failure Definition:}
\CheckBox[name=faildefload,charsize=12pt]{Failure definition(s) in test database loads into list view.}

\subsubsection{Edit Actions:}
\CheckBox[name=failidnoedit,charsize=12pt]{Failure ID can \textit{not} be edited.} \\
\CheckBox[name=faildefedit,charsize=12pt]{Failure definition can be edited and edits persist.}

\subsubsection{Insert Actions:}
\CheckBox[name=faildefadd,charsize=12pt]{A new failure definition can be added using the 'Insert Sibling' button in the list view buttonbox.}

\subsubsection{Delete Actions:}
\CheckBox[name=faildefdelete,charsize=12pt]{The selected failure definition can be deleted using the 'Remove' button in the list view buttonbox.}

\subsubsection{Save Actions:}
\CheckBox[name=faildefsave,charsize=12pt]{Edit to the selected failure definition can be saved to the database using the 'Save' button in the list view buttonbox.} \\
\CheckBox[name=faildefadd,charsize=12pt]{Edits to all failure definitions can be saved to the database using the 'Save All' button in the list view buttonbox.} \\

\noindent Submit your results or reset this form: \\

\Submit{Submit}\hspace{50pt} \Reset{Clear}
\hfill ~\\
\end{Form}

\end{document}
