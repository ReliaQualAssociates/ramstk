\documentclass[twoside,12pt,letterpaper,openright]{book}

\usepackage{dirtree}
\usepackage{listings}

\begin{document}


\title{The Reliability ToolKit (RTK) User's Guide}

% -----------------------------------------------------------------------------
% -----------------------------------------------------------------------------
\chapter{Installing RTK}

\noindent The Reliability ToolKit (RTK) is available via PyPi and can be 
installed on a Linux machine using one of the following commands:

\begin{lstlisting}
    $ pip install RTK
\end{lstlisting}

\noindent or (deprecated and not recommended)

\begin{lstlisting}
    $ easy_install RTK
\end{lstlisting}

\noindent On Windows, the exe or msi installer must be downloaded and executed.
\\

\noindent Once the RTK application is installed, it may be launched from the 
desktop icon, menu entry, or a shell.  On first launch, RTK should detect first 
run status, setup some user-specific configurations, and copy system-wide 
configuration files into the user's profile.  Simply follow the prompts, 
selecting default options if unsure.

% -----------------------------------------------------------------------------
% Thus begins a brief description of RTK.
% -----------------------------------------------------------------------------
\chapter{Getting Started}

\section{The RTK Conceptual Layout}

\noindent RTK is a multi-user, database-driven solution for reliability 
program management and reliability analysis.  The use of a database provides 
greater potential to integrate reliability program activities and tools, while 
facilitating effective information sharing between engineers and engineering
teams.  Databases also provide greater security, auditability, and data 
fidelity than the use of stand alone, single user, file format solutions.
\\

\noindent However, for users making the transition from spreadsheets and other 
single file soulations, there will be a learning curve.  ReliaQual Associates, 
LLC doesn't believe the learning curve is steep, but some discussion of the 
design of RTK and terminology is in order to orient the new user.
\\

\noindent RTK works from a hierarchical or tree concept.  At the top of the 
hierarchy is the RTK Program.  The RTK Program represents the database storing 
the analysis data and results.  Each reliability program within an organization 
would have it's own database and, thus, it's own RTK Program. \\

\noindent At the next level in the hierarchy, an RTK Program is comprised of 
modules (For those familiar with databases, a module represents a view of a 
single table or a JOIN query of multiple tables with related information.)  
There are currently five modules available for RTK.  These are:
\\

\begin{itemize}
    \item Revision
    \item Function
    \item Requirement
    \item Hardware
    \item Validation
\end{itemize}

\noindent Every RTK Program must have a Revision module; the remainder are 
optional.
\\

\noindent Each module groups related data and analyses.  With the exception of 
the Revision module, all the data and analyses are related to a single module.  
An RTK Program can have multiple Revision modules, but a Revision module can 
only have one of each subordinate module.  Visually, the hierarchy of RTK is:
\\

\dirtree{%
.1 RTK Program.
.2 Revision -.
.3 Function (for Revision -).
.3 Requirement (for Revision -).
.3 Hardware (for Revision -).
.3 Validation (for Revision -).
.2 Revision A.
.3 Function (for Revision A).
.3 Requirement (for Revision A).
.3 Hardware (for Revision A).
.3 Validation (for Revision A).
}

\medskip
\noindent Most RTK modules will have additional information and analyses 
associated with them.  Information and analyses related to each of the modules
are:

\begin{itemize}
    \item Revision
        \begin{itemize}
            \item Usage Profile
            \item Failure Definition
        \end{itemize}
    \item Function
        \begin{itemize}
            \item Design Failure Mode and Effects Analysis (DFMEA)
        \end{itemize}
    \item Requirement
        \begin{itemize}
            \item Stakeholder Input
        \end{itemize}
    \item Hardware
        \begin{itemize}
            \item Allocation
            \item Hazard Analysis
            \item Similar Item Analysis
            \item Reliability Assessment (Prediction)
            \item (Design) Failure Mode, Effects, and (Criticality) Analysis (D)FME(C)A
            \item Physics of Failure Analysis/Damage Modeling
        \end{itemize}
    \item Validation
\end{itemize}

\noindent As you can see, the type of information and analyses under each RTK 
module are related to that module in their scope.  For example, 

\begin{quote}
Failure definitions are related to the Revision module because failures should 
be defined early for the entire program and are applicable to all modules as 
they are applicable throughout all phases of the program.
\end{quote}

\begin{quote}
Reliability allocations, hazards analysis, etc. are related to the Hardware
module because they are analyzing the Hardware design.
\end{quote}

\noindent Visually, including the information and analyses associated with each
module, the hierarchy of RTK is:
\\

\dirtree{%
.1 RTK Program.
.2 Revision -.
.2 Usage Profile.
.2 Failure Definition.
.3 Function (for Revision -).
.4 Design Failure Mode and Effects Analysis (DFMEA).
.3 Requirement (for Revision -).
.4 Stakeholder Inputs.
.3 Hardware (for Revision -).
.4 Allocation.
.4 Hazard Analysis (HazOps).
.4 Similar Item Analysis.
.4 Reliability Assessment (Prediction).
.4 (Design) Failure Mode, Effects, and (Criticality) Analysis (D)FME(C)A.
.4 Physics of Failure Analysis/Damage Modeling.
.3 Validation (for Revision -).
}

\section{The RTK Visual Layout}

\noindent The RTK application is a graphical user interface (GUI) application
only.  Designed for use primarily on workstations using multiple monitors, the
RTK GUI is presented as three windows.  Each window presents a particular set
of information or an analysis although some data is shared between windows when
appropriate.

\noindent Each of the three GUI windows is referred to as a 'book'.  Books 
organize data and analyses.  With multiple books, users can place lesser used
or supporting data on a secondary monitor while keeping the bulk of their work
on the primary monitor.  The three RTK books are:

\begin{enumerate}
    \item Module Book
    \item Work Book
    \item Lists and Matrices Book
\end{enumerate}

\subsection{The Module Book}

\noindent The Module Book, shown in \ref{figure1}, groups all of the RTK 
modules.  This book has a page for each of the RTK modules activated in the 
open RTK Program.  These tabs, left to right, generally follow the flow of a 
development program.

\noindent In the Revision (first) page, the user would select the Revision
they are interested in working with.  This causes the information related to
the selected Revision for the other RTK modules to be loaded.  Thus, the 
Functions listed on the Function page in the Module Book are those functions
related to the selected Revision only.

\noindent Some RTK modules display module information in a flat list (Revision
and Validation).  Others display module information in a hierachical list 
(Function, Requirement, and Hardware).

\subsection{The Work Book}

\noindent The Work Book, shown in \ref{figure2}, is where the bulk of the data
entry and analyses in RTK takes place.  The information and analyses displayed 
in the Work Book is the information and analyses associated with the line item 
selected in the Module Book.  For example, if the Hardware page is selected in 
the Module Book, the information and analyses shown in the Work Book are 
associated with the line selected in the Module Book's Hardware page.

\noindent All RTK Modules will have a General Data page in their Work Book.  
Each module will have other Work Book pages as appropriate for the type of 
information and analyses related to them.  

\noindent In the Work Book, text fields that accept user input will be 
displayed with a white background using normal weight font.  Text fields that
display calculated results will be displayed with a light blue background using
bold font.  The background color is a user-specific option and may be changed.

\subsection{The Lists and Matrices Book}

\noindent The Lists and Matrices Book, shown in \ref{figure3}, contains the 
supporting information for the selected RTK module.  Lists may be flat such as
Failure Definitions or hierarchical such as the Usage Profile.

\noindent The Matrices are used to show relationships between the selected RTK
Module and other RTK Modules.  Matrices will show the items in the selected RTK
Module along the left side (rows).  The other RTK Module items are listed along
the top (columns).  In the intersection of the row and column, the user may
select 'Partial' or 'Complete' to indicate the strength of the relationship.
Selecting nothing would indicate a lack of relationship.  Each Matrix will have
its own page in the Lists and Matrices Book and there may be multiple matrices
for a selected RTK Module.

\noindent For example, if the Module Book Function page is selected, the Lists
and Matrices Book will have a page displaying Functions along the left side 
(rows) and Verification tasks along the top (columns).  Selecting 'Partial' in 
the intersection of a Function and Verification task indicates the Verfication 
task partially verifies the function while selecting 'Complete' indicates the
task completely verifies the function.  If nothing is selected, this would 
indicate a lack of a Verification task for that Function.  Cells with no 
relationship are shown in white, partial cells are shown in pink, and complete 
cells are shown in green.  This provides a quick visual representation of the 
Verification plan as it is related to the system's Functions.

\noindent The 

\begin{itemize}
    \item Function
        \begin{itemize}
            \item Function-Requirement
            \item Function-Hardware
            \item Function-Validation
        \end{itemize}
    \item Requirement
        \begin{itemize}
            \item Requirement-Hardware
            \item Requirement-Validation
        \end{itemize}
    \item Hardware
        \begin{itemize}
            \item Hardware-Validation
        \end{itemize}
\end{itemize}


% -----------------------------------------------------------------------------
% Thus begins the description of the RTK modules in great detail.
% -----------------------------------------------------------------------------
\chapter{The RTK Modules}

\noindent This chapter describes each of the RTK modules in detail.

\section{The Revision Module}

\noindent The Revision Module is the only RTK module \textbf{required} to be 
used.  All other RTK modules are optional and are relative to the Revision 
module.
\\

\noindent A Revision could be used to represent many things depending on your industry.  A Revision could be a:

\begin{itemize}
	\item {Model Year}
	\item {Configuration}
	\item {Variant}
\end{itemize}

\noindent As mentioned above, the Revision Module Work Book only has a General 
Data page where the following information is available for editing for the 
selected Revision:

\begin{itemize}
    \item Name
    \item Description
    \item Remarks
\end{itemize}

\noindent There are no analyses associated with the Revision Module.  Support
information associated with a Revision includes Usage Profiles and Failure
Definitions.  Each is discussed in greater detail in the next two sections.
\\

\subsection{Usage Profile}

\noindent The Usage Profile describes the tasks, durations, and environments in
which the system being developed is expected to perform each mission.  Without 
a Usage Profile, reliability specifications and analyses have no real meaning.


\noindent The Usage Profile(s) should be defined and documented early in the 
development program.  Each mission can be broken down into one or more mission 
phase.  For each mission phase, one or more environmental conditions can be 
defined.  A simple usage profile, for example, might be:
\\

\dirtree{%
.1 Drive to work (Mission).
.2 Start car (Mission Phase).
.3 Temperature, Ambient (Environment).
.3 Humidity.
.2 Transit from home to work (Mission Phase).
.3 Temperature, Ambient.
.3 Humidity.
.3 Precipitation.
.3 Vibration.
.3 Shock.
.2 Stop car (Mission Phase).
.3 Temperature, Ambient.
.3 Humidity.
.3 Precipitation.
}

\medskip
\noindent In the Usage Profile module, the following information is editable
for each entity:
\\

\begin{tabular}{l | l}
    \hline \hline
    \textbf{Entity} & \textbf{Editable Information} \\
    \hline
    Mission & Mission description \\
            & Mission start time \\
            & Mission end time \\
    \cline{2-2}
    Mission Phase & Mission phase code \\
                  & Mission phase description \\
                  & Mission phase start time \\
                  & Mission phase end time \\
    \cline{2-2}
    Environment & Environmental condition description \\
                & Environmental condition measurement units \\
                & Minimum design value \\
                & Maximum design value \\
                & Mean design value \\
                & Variance of design value \\
    \hline
\end{tabular}

\medskip
\noindent In addition to defining the usage profile, the mission and mission 
phase will be used in Hardware FMECA's to calculate a failure mode's mission 
time.

\subsection{Failure Definitions}

\noindent Failure definitions should be developed and agreed upon early in the 
development program.  These failure definitions should be used throughout the
entire life-cycle of the product.  It is best to define failures as the 
functions are being defined.  Functional failure definitions will fall into one
of the following categories:

\begin{enumerate}
    \item Too much function.
    \item Too little function.
    \item Intermittent functionality.
    \item Function not there when required.
    \item Function present when not required.
\end{enumerate}

\noindent As requirements/specifications are identified these functional 
failure definitions can be amended with performance values or new, 
performance-based failure definitions can be added.

\section{The Function Module}
\subsection{Design Failure Mode and Effects Analysis [DFMEA]}
\section{The Requirement Module}
\subsection{Stakeholder Inputs}
\section{The Hardware Module}

Incidents
Survival Analyses
Hardware-Test
Hardware-Validation

\subsection{Reliability Allocation}
\subsection{Hazards Analysis (HazOp)}
\subsection{Similar Item Analysis}
\subsection{Reliability Assessment}
\subsection{(Design) Failure Mode, Effects, (and Criticality) Analysis [(D)FME(C)A]}
\subsubsection{Criticality Analysis}

\noindent RTK supports both the MIL-STD-1629A, Task 102 approach to risk 
categorization as well as the risk priority number (RPN) approach.

\subsection{Physics of Failure Analysis}
\section{The Validation Module}

\end{document}