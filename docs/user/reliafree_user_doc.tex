\documentclass[11pt, 12pt, twoside, onecolumn]{book}

\usepackage[english]{babel} %language selection
\selectlanguage{english}

\usepackage{amsmath}    	% Include the AMS Mathematics package.
\usepackage{graphicx}     	% Include the graphics package.
\usepackage{hyperref}      	% Include the hyper reference package.

% Setup the format of hyperlinks.
\hypersetup{colorlinks, citecolor=black, filecolor=black, 
			linkcolor=black, urlcolor=blue, bookmarksopen=true, 
			pdftex}

% Set the page-styles.
\pdfpagewidth 8.5in
\pdfpageheight 11in
\pagestyle{plain}
\pagenumbering{arabic}

% Unique commands used in this document.
\newcommand{\code}[1]{\textrm{\textit{#1}}}

\begin{document}

\begin{titlepage}
\centering
{\linespread{1.3} \Huge \textbf{The ReliaFree User Manual}} \\
{\linespread{1.3} \large \textbf{Andrew Rowland}} \\
{\large \textbf{Date}}
\end{titlepage}

\tableofcontents
\newpage

\chapter{\textbf{Overview}}

\noindent The ReliaFree application is intended to be an open-source, free replacement for proprietary reliability, availability, and maintainability (RAMS) programs.  ReliaFree is capable of assisting in the performance and management of the following RAMS tasks:

	\begin{itemize}
		\item Reliability Assessments (Predictions)
		\item Failure Mode, Effects, and Criticality Analysis (FMEA/FMECA)
		\item Failure Reporting, Analysis, and Corrective Action (FRACA)
	\end{itemize}

\noindent Figure~\ref{application\_first\_open} shows the view presented to the user when ReliaFree is first launched.  There are three windows, each of which is independent in screen placement.  That is, the windows can be placed wherever convenient for the user.  The data in each of these windows is, however, linked to the data in the other windows.

	\begin{figure}[ht!]
		\centering
		\includegraphics[width=1.0\textwidth]{figures/application_first_open.png}
		\caption{The ReliaFree Application When First Opened}\label{application\_first\_open}
	\end{figure}

\noindent The upper window is referred to as the ``Program Window.''  This window is used to display information relative to the development program being analyzed.  Various program-related information is presented in a notebook of gtk.Treeviews.  The tabs in the ``Program Notebook'' are:

	\begin{itemize}
		\item Revision Tree
		\item Function Tree
		\item Hardware Tree
		\item FMECA Tree
		\item FRACA Tree
	\end{itemize}

\noindent Figure~\ref{program\_first\_open} shows ReliaFree when a program database is first opened.

	\begin{figure}[ht!]
		\centering
		\includegraphics[width=1.0\textwidth]{figures/application_first_open.png}
		\caption{The ReliaFree Application When Program Database is First Opened}\label{program\_first\_open}
	\end{figure}

\section{\textbf{Revision Tree Overview}}

\noindent 

\chapter{\textbf{Working with a Project}}

\section{\textbf{Creating the Project}}

\noindent A new project can be created either from the menu bar or the toolbar.  To create a new project:

\begin{itemize}
	\item Select the \textunderscore{F}ile menu or click the New icon on the toolbar.
	\item If using the File menu, select New,
	\item Select Project and a dialog box similar to the one show in Figure appears,
	\item Enter the data requested by the dialog box,
	\item Click the OK button to create the Project or the Cancel button to cancel the operation.
\end{itemize}

\noindent Once you click OK, a SQL script (new\_program.sql) is executed to create a MySQL database with the name you provided in step 4 above.  This database is then opened by ReliaFree available to work with.

\section{\textbf{Opening a Project}}

\noindent
\section{\textbf{Adding Items to the Project}}

\noindent Once you
\end{document}