%
% API Documentation for ReliaFRee
% Package reliafree
%
% Generated by epydoc 3.0.1
% [Fri Apr 13 09:40:38 2012]
%

%%%%%%%%%%%%%%%%%%%%%%%%%%%%%%%%%%%%%%%%%%%%%%%%%%%%%%%%%%%%%%%%%%%%%%%%%%%
%%                          Module Description                           %%
%%%%%%%%%%%%%%%%%%%%%%%%%%%%%%%%%%%%%%%%%%%%%%%%%%%%%%%%%%%%%%%%%%%%%%%%%%%

    \index{reliafree \textit{(package)}|(}
\section{Package reliafree}

    \label{reliafree}

%%%%%%%%%%%%%%%%%%%%%%%%%%%%%%%%%%%%%%%%%%%%%%%%%%%%%%%%%%%%%%%%%%%%%%%%%%%
%%                                Modules                                %%
%%%%%%%%%%%%%%%%%%%%%%%%%%%%%%%%%%%%%%%%%%%%%%%%%%%%%%%%%%%%%%%%%%%%%%%%%%%

\subsection{Modules}

\begin{itemize}
\setlength{\parskip}{0ex}
\item \textbf{assembly}: This is the Class that is used to represent and hold information related to
the hardware assemblies of the Program.



  \textit{(Section \ref{reliafree:assembly}, p.~\pageref{reliafree:assembly})}

\item \textbf{calculations}: This file contains various calculations used by the RelKit Project.



  \textit{(Section \ref{reliafree:calculations}, p.~\pageref{reliafree:calculations})}

\item \textbf{capacitors}
  \textit{(Section \ref{reliafree:capacitors}, p.~\pageref{reliafree:capacitors})}

  \begin{itemize}
\setlength{\parskip}{0ex}
    \item \textbf{capacitor}: Capacitor is the meta class for all capacitor types.



  \textit{(Section \ref{reliafree:capacitors:capacitor}, p.~\pageref{reliafree:capacitors:capacitor})}

    \item \textbf{electrolytic}
  \textit{(Section \ref{reliafree:capacitors:electrolytic}, p.~\pageref{reliafree:capacitors:electrolytic})}

    \item \textbf{fixed}
  \textit{(Section \ref{reliafree:capacitors:fixed}, p.~\pageref{reliafree:capacitors:fixed})}

    \item \textbf{variable}
  \textit{(Section \ref{reliafree:capacitors:variable}, p.~\pageref{reliafree:capacitors:variable})}

  \end{itemize}
\item \textbf{component}: This is the Class that is used to represent and hold information related to
the hardware components of the Program.



  \textit{(Section \ref{reliafree:component}, p.~\pageref{reliafree:component})}

\item \textbf{configuration}: This file contains configuration information and functions for RelKit.



  \textit{(Section \ref{reliafree:configuration}, p.~\pageref{reliafree:configuration})}

\item \textbf{connections}
  \textit{(Section \ref{reliafree:connections}, p.~\pageref{reliafree:connections})}

  \begin{itemize}
\setlength{\parskip}{0ex}
    \item \textbf{connection}
  \textit{(Section \ref{reliafree:connections:connection}, p.~\pageref{reliafree:connections:connection})}

    \item \textbf{multipin}
  \textit{(Section \ref{reliafree:connections:multipin}, p.~\pageref{reliafree:connections:multipin})}

    \item \textbf{pcb}
  \textit{(Section \ref{reliafree:connections:pcb}, p.~\pageref{reliafree:connections:pcb})}

    \item \textbf{socket}
  \textit{(Section \ref{reliafree:connections:socket}, p.~\pageref{reliafree:connections:socket})}

    \item \textbf{solder}
  \textit{(Section \ref{reliafree:connections:solder}, p.~\pageref{reliafree:connections:solder})}

  \end{itemize}
\item \textbf{function}: This is the Class that is used to represent and hold information related to
the functions of the Program.



  \textit{(Section \ref{reliafree:function}, p.~\pageref{reliafree:function})}

\item \textbf{hardware}: This is the Class that is used to represent and hold information related to
the assemblies of the Program.



  \textit{(Section \ref{reliafree:hardware}, p.~\pageref{reliafree:hardware})}

\item \textbf{inductors}
  \textit{(Section \ref{reliafree:inductors}, p.~\pageref{reliafree:inductors})}

  \begin{itemize}
\setlength{\parskip}{0ex}
    \item \textbf{coil}
  \textit{(Section \ref{reliafree:inductors:coil}, p.~\pageref{reliafree:inductors:coil})}

    \item \textbf{inductor}
  \textit{(Section \ref{reliafree:inductors:inductor}, p.~\pageref{reliafree:inductors:inductor})}

    \item \textbf{transformer}
  \textit{(Section \ref{reliafree:inductors:transformer}, p.~\pageref{reliafree:inductors:transformer})}

  \end{itemize}
\item \textbf{integrated\_circuits}
  \textit{(Section \ref{reliafree:integrated_circuits}, p.~\pageref{reliafree:integrated_circuits})}

  \begin{itemize}
\setlength{\parskip}{0ex}
    \item \textbf{gaas}
  \textit{(Section \ref{reliafree:integrated_circuits:gaas}, p.~\pageref{reliafree:integrated_circuits:gaas})}

    \item \textbf{ic}
  \textit{(Section \ref{reliafree:integrated_circuits:ic}, p.~\pageref{reliafree:integrated_circuits:ic})}

    \item \textbf{linear}
  \textit{(Section \ref{reliafree:integrated_circuits:linear}, p.~\pageref{reliafree:integrated_circuits:linear})}

    \item \textbf{logic}
  \textit{(Section \ref{reliafree:integrated_circuits:logic}, p.~\pageref{reliafree:integrated_circuits:logic})}

    \item \textbf{memory}
  \textit{(Section \ref{reliafree:integrated_circuits:memory}, p.~\pageref{reliafree:integrated_circuits:memory})}

    \item \textbf{microprocessor}
  \textit{(Section \ref{reliafree:integrated_circuits:microprocessor}, p.~\pageref{reliafree:integrated_circuits:microprocessor})}

    \item \textbf{palpla}
  \textit{(Section \ref{reliafree:integrated_circuits:palpla}, p.~\pageref{reliafree:integrated_circuits:palpla})}

    \item \textbf{vlsi}
  \textit{(Section \ref{reliafree:integrated_circuits:vlsi}, p.~\pageref{reliafree:integrated_circuits:vlsi})}

  \end{itemize}
\item \textbf{login}
  \textit{(Section \ref{reliafree:login}, p.~\pageref{reliafree:login})}

\item \textbf{main}: This is the main program for The RelKit application.



  \textit{(Section \ref{reliafree:main}, p.~\pageref{reliafree:main})}

\item \textbf{meters}
  \textit{(Section \ref{reliafree:meters}, p.~\pageref{reliafree:meters})}

  \begin{itemize}
\setlength{\parskip}{0ex}
    \item \textbf{meter}
  \textit{(Section \ref{reliafree:meters:meter}, p.~\pageref{reliafree:meters:meter})}

  \end{itemize}
\item \textbf{miscellaneous}
  \textit{(Section \ref{reliafree:miscellaneous}, p.~\pageref{reliafree:miscellaneous})}

  \begin{itemize}
\setlength{\parskip}{0ex}
    \item \textbf{crystal}
  \textit{(Section \ref{reliafree:miscellaneous:crystal}, p.~\pageref{reliafree:miscellaneous:crystal})}

    \item \textbf{filter}
  \textit{(Section \ref{reliafree:miscellaneous:filter}, p.~\pageref{reliafree:miscellaneous:filter})}

    \item \textbf{fuse}
  \textit{(Section \ref{reliafree:miscellaneous:fuse}, p.~\pageref{reliafree:miscellaneous:fuse})}

    \item \textbf{lamp}
  \textit{(Section \ref{reliafree:miscellaneous:lamp}, p.~\pageref{reliafree:miscellaneous:lamp})}

  \end{itemize}
\item \textbf{mysql}
  \textit{(Section \ref{reliafree:mysql}, p.~\pageref{reliafree:mysql})}

\item \textbf{notebook}: This is the Workbook window for RelKit.



  \textit{(Section \ref{reliafree:notebook}, p.~\pageref{reliafree:notebook})}

\item \textbf{partlist}: This is the Parts List window for RelKit.



  \textit{(Section \ref{reliafree:partlist}, p.~\pageref{reliafree:partlist})}

\item \textbf{relays}
  \textit{(Section \ref{reliafree:relays}, p.~\pageref{reliafree:relays})}

  \begin{itemize}
\setlength{\parskip}{0ex}
    \item \textbf{relay}
  \textit{(Section \ref{reliafree:relays:relay}, p.~\pageref{reliafree:relays:relay})}

  \end{itemize}
\item \textbf{requirement}: This is the Class that is used to represent and hold information related to
the requirements of the Program.



  \textit{(Section \ref{reliafree:requirement}, p.~\pageref{reliafree:requirement})}

\item \textbf{resistors}
  \textit{(Section \ref{reliafree:resistors}, p.~\pageref{reliafree:resistors})}

  \begin{itemize}
\setlength{\parskip}{0ex}
    \item \textbf{fixed}
  \textit{(Section \ref{reliafree:resistors:fixed}, p.~\pageref{reliafree:resistors:fixed})}

    \item \textbf{resistor}
  \textit{(Section \ref{reliafree:resistors:resistor}, p.~\pageref{reliafree:resistors:resistor})}

    \item \textbf{thermistor}
  \textit{(Section \ref{reliafree:resistors:thermistor}, p.~\pageref{reliafree:resistors:thermistor})}

    \item \textbf{variable}
  \textit{(Section \ref{reliafree:resistors:variable}, p.~\pageref{reliafree:resistors:variable})}

  \end{itemize}
\item \textbf{revision}: This is the Class that is used to represent and hold information related to
the revision of the Program.



  \textit{(Section \ref{reliafree:revision}, p.~\pageref{reliafree:revision})}

\item \textbf{semiconductors}
  \textit{(Section \ref{reliafree:semiconductors}, p.~\pageref{reliafree:semiconductors})}

  \begin{itemize}
\setlength{\parskip}{0ex}
    \item \textbf{diode}
  \textit{(Section \ref{reliafree:semiconductors:diode}, p.~\pageref{reliafree:semiconductors:diode})}

    \item \textbf{optoelectronics}
  \textit{(Section \ref{reliafree:semiconductors:optoelectronics}, p.~\pageref{reliafree:semiconductors:optoelectronics})}

    \item \textbf{semiconductor}
  \textit{(Section \ref{reliafree:semiconductors:semiconductor}, p.~\pageref{reliafree:semiconductors:semiconductor})}

    \item \textbf{thyristor}
  \textit{(Section \ref{reliafree:semiconductors:thyristor}, p.~\pageref{reliafree:semiconductors:thyristor})}

    \item \textbf{transistor}
  \textit{(Section \ref{reliafree:semiconductors:transistor}, p.~\pageref{reliafree:semiconductors:transistor})}

  \end{itemize}
\item \textbf{switches}
  \textit{(Section \ref{reliafree:switches}, p.~\pageref{reliafree:switches})}

  \begin{itemize}
\setlength{\parskip}{0ex}
    \item \textbf{breaker}
  \textit{(Section \ref{reliafree:switches:breaker}, p.~\pageref{reliafree:switches:breaker})}

    \item \textbf{rotary}
  \textit{(Section \ref{reliafree:switches:rotary}, p.~\pageref{reliafree:switches:rotary})}

    \item \textbf{sensitive}
  \textit{(Section \ref{reliafree:switches:sensitive}, p.~\pageref{reliafree:switches:sensitive})}

    \item \textbf{switch}
  \textit{(Section \ref{reliafree:switches:switch}, p.~\pageref{reliafree:switches:switch})}

    \item \textbf{thumbwheel}
  \textit{(Section \ref{reliafree:switches:thumbwheel}, p.~\pageref{reliafree:switches:thumbwheel})}

    \item \textbf{toggle}
  \textit{(Section \ref{reliafree:switches:toggle}, p.~\pageref{reliafree:switches:toggle})}

  \end{itemize}
\item \textbf{tree}: This is the System Tree window for RelKit.



  \textit{(Section \ref{reliafree:tree}, p.~\pageref{reliafree:tree})}

\item \textbf{utilities}: utilities contains utility functions for interacting with the RelKit 
application.



  \textit{(Section \ref{reliafree:utilities}, p.~\pageref{reliafree:utilities})}

\item \textbf{validation}: This is the Class that is used to represent and hold information related to
verification and validation tasks of the Program.



  \textit{(Section \ref{reliafree:validation}, p.~\pageref{reliafree:validation})}

\item \textbf{widgets}: widgets contains functions for creating, populating, destroying, and 
interacting with pyGTK widgets.



  \textit{(Section \ref{reliafree:widgets}, p.~\pageref{reliafree:widgets})}

\end{itemize}


%%%%%%%%%%%%%%%%%%%%%%%%%%%%%%%%%%%%%%%%%%%%%%%%%%%%%%%%%%%%%%%%%%%%%%%%%%%
%%                               Variables                               %%
%%%%%%%%%%%%%%%%%%%%%%%%%%%%%%%%%%%%%%%%%%%%%%%%%%%%%%%%%%%%%%%%%%%%%%%%%%%

  \subsection{Variables}

    \vspace{-1cm}
\hspace{\varindent}\begin{longtable}{|p{\varnamewidth}|p{\vardescrwidth}|l}
\cline{1-2}
\cline{1-2} \centering \textbf{Name} & \centering \textbf{Description}& \\
\cline{1-2}
\endhead\cline{1-2}\multicolumn{3}{r}{\small\textit{continued on next page}}\\\endfoot\cline{1-2}
\endlastfoot\raggedright \_\-\_\-p\-a\-c\-k\-a\-g\-e\-\_\-\_\- & \raggedright \textbf{Value:} 
{\tt \texttt{'}\texttt{reliafree}\texttt{'}}&\\
\cline{1-2}
\end{longtable}

    \index{reliafree \textit{(package)}|)}
