%
% API Documentation for ReliaFRee
% Module reliafree.assembly
%
% Generated by epydoc 3.0.1
% [Fri Apr 13 09:40:38 2012]
%

%%%%%%%%%%%%%%%%%%%%%%%%%%%%%%%%%%%%%%%%%%%%%%%%%%%%%%%%%%%%%%%%%%%%%%%%%%%
%%                          Module Description                           %%
%%%%%%%%%%%%%%%%%%%%%%%%%%%%%%%%%%%%%%%%%%%%%%%%%%%%%%%%%%%%%%%%%%%%%%%%%%%

    \index{reliafree \textit{(package)}!reliafree.assembly \textit{(module)}|(}
\section{Module reliafree.assembly}

    \label{reliafree:assembly}
This is the Class that is used to represent and hold information related to
the hardware assemblies of the Program.


%%%%%%%%%%%%%%%%%%%%%%%%%%%%%%%%%%%%%%%%%%%%%%%%%%%%%%%%%%%%%%%%%%%%%%%%%%%
%%                                Classes                                %%
%%%%%%%%%%%%%%%%%%%%%%%%%%%%%%%%%%%%%%%%%%%%%%%%%%%%%%%%%%%%%%%%%%%%%%%%%%%

\subsection{Classes}

\begin{itemize}  \setlength{\parskip}{0ex}
  \item \textbf{Assembly}: The Assembly class is used to represent a piece of hardware in a system 
being analyzed.



  \textit{(Section \ref{reliafree:assembly:Assembly}, p.~\pageref{reliafree:assembly:Assembly})}

\end{itemize}

%%%%%%%%%%%%%%%%%%%%%%%%%%%%%%%%%%%%%%%%%%%%%%%%%%%%%%%%%%%%%%%%%%%%%%%%%%%
%%                               Variables                               %%
%%%%%%%%%%%%%%%%%%%%%%%%%%%%%%%%%%%%%%%%%%%%%%%%%%%%%%%%%%%%%%%%%%%%%%%%%%%

  \subsection{Variables}

    \vspace{-1cm}
\hspace{\varindent}\begin{longtable}{|p{\varnamewidth}|p{\vardescrwidth}|l}
\cline{1-2}
\cline{1-2} \centering \textbf{Name} & \centering \textbf{Description}& \\
\cline{1-2}
\endhead\cline{1-2}\multicolumn{3}{r}{\small\textit{continued on next page}}\\\endfoot\cline{1-2}
\endlastfoot\raggedright \_\-\_\-p\-a\-c\-k\-a\-g\-e\-\_\-\_\- & \raggedright \textbf{Value:} 
{\tt \texttt{'}\texttt{reliafree}\texttt{'}}&\\
\cline{1-2}
\end{longtable}

    \index{reliafree \textit{(package)}!reliafree.assembly \textit{(module)}|)}
