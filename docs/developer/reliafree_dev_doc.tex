\documentclass[11pt, 12pt, twoside, onecolumn]{article}

%%%%%%%%%%%%%%%%%%%%%%%%%%%%%%
% Set included packages
\usepackage[english]{babel} %language selection
\selectlanguage{english}

\usepackage{amsmath}        % Include the AMS Mathematics package.
\usepackage{graphicx}       % Include the graphics package.
\usepackage{hyperref}       % Include the hyper reference package.
\usepackage{tabularx}       % Allows multi-row columns
\usepackage{longtable}      % Allows big tables.
\usepackage{lscape}         % Allows landscape mode.
%%%%%%%%%%%%%%%%%%%%%%%%%%%%%%

%%%%%%%%%%%%%%%%%%%%%%%%%%%%%%
% Setup the format of hyperlinks.
\hypersetup{colorlinks, citecolor=black,
            filecolor=black, linkcolor=black,
            urlcolor=blue, bookmarksopen=true,
            pdftex}
%%%%%%%%%%%%%%%%%%%%%%%%%%%%%%

%%%%%%%%%%%%%%%%%%%%%%%%%%%%%%
% Set the page-styles.
\pagestyle{plain}
\pagenumbering{arabic}
\pdfpagewidth   8.5in
\pdfpageheight  11in
\oddsidemargin  0.0in
\evensidemargin 0.0in
\textwidth      6.5in
\headheight     0.0in
\topmargin      0.0in
\textheight     9.0in
%%%%%%%%%%%%%%%%%%%%%%%%%%%%%%

%%%%%%%%%%%%%%%%%%%%%%%%%%%%%%
% Unique commands used in this document.
\newcommand{\code}[1]{\textrm{\textit{#1}}}
%%%%%%%%%%%%%%%%%%%%%%%%%%%%%%

\begin{document}

\begin{titlepage}
\centering
{\Large \bf Developer Information} \\
{\Large \bf for} \\
{\Large \bf The RTK Project} \\
\bigskip
{\large \bf Andrew "Weibullguy" Rowland} \\
{\Large \bf date}
\end{titlepage}
\linespread{1}

\tableofcontents
\newpage

\section{\bf \Large The RTK Project Overview}

\noindent The RTK Project aims to provide a FOSS alternative to proprietary reliability, availability, maintainability, and safety (RAMS) analysis applications.  As a FOSS alternative, proprietary models such as PRISM or 217Plus can not be incorporated in the distributed package because The RTK Project does not have the authority or the rights to distribute them.  As such, only models that can be found in the public domain can be incorporated. \\

\noindent At this time, The RTK Project is working towards implementing MIL-HDBK-217F, Notice 2 reliability prediction models.  It is expected that future development will include the structure to accomodate such models as PRISM and 217Plus.  However, it will be up to an individual user to populate the structure with the pertinent data to actually implement these models. \\

\noindent The RTK Project is also working towards implementing MIL-STD-1629A Failure Mode and Effects Analysis (FMEA) and Criticality Analysis (CA or FMECA when combined with the FMEA).  Risk Priority Number (RPN) methods will also be implemented by The RTK Project. \\

\subsection{\bf \large Definitions}

\noindent There are some terms used by The RTK Project that may be ambiguous to a software developer or others not familiar with RAMS.  These are defined below.

\begin{enumerate}
    \item \textbf{Hazard Rate} - this is the mathematically correct terminology for the layman's term failure rate.
    \item \textbf{Program} - this is a contractual program, not a "computer program."  The RTK Project refers to a program when discussing a system under analysis.
    \item \textbf{Reliability} - the probability of failure free operation for a specified period of time in a specified environment.
\end{enumerate}

\section{\bf \Large RTK Program Databases}

\noindent The RTK Project currently uses MySQL databases to store information.  This includes program-specific information as well as information common to all programs.  There will be one MySQL database per program.  Additionally, there will be one common database for a given installation of The RTK Project.  It is recommended that a dedicated MySQL server be used for The RTK Project.  This isn't strictly necessary and, as a developer, you may not have the resources to dedicate a machine for this purpose.  Bear in mind, however, that the recommendation is to dedicate a MySQL server to The RTK Project, NOT an entire machine.  Therefore, the machine with the dedicate MySQL server can be used for other purposes simultaneously. \\

\noindent It is also possible to have multiple MySQL servers that The RTK Project is capable of using.  There may be very good reasons for configuring things this way.  For example, individuals working on Program A may not be authorized to view information related to Program B.  In this case, Program A can be created in one server and Program B created on another.  Program A individuals will not be granted access to the Program B server. \\

\noindent Unless stated otherwise for specific fields, default values for each data type shall be:

\bigskip
\begin{tabular}{l c}
float & 1.0 \\
int & 0 \\
str & ``'' \\
varchar & Null \\
\end{tabular}
\bigskip

\noindent As discussed above, there is some information that will be common to all programs.  This includes such things as types of prediction models available, model input data, and lists of supported probability distributions.  All this common information is stored in a common database (reliafreecom) to prevent duplication of data that would then need to be kept synchronized.  This common database can be kept on all MySQL servers providing The RTK Project program databases or it can be kept on a central server.  The RTK Project recommends a central server to ensure updates to the common database are seen by all program databases. \\

\noindent Common information is just that, common.  It is not changeable by the user and login information, including password, is hard-coded in the site-wide configuration file.  This does reduce security, but not as much as providing this login information to every person using The RTK Project at a given site.  Since this information is stored in the site-wide configuration file, the user and password can be customized for a site. \\

\noindent As a developer, you can find two *.sql files in the configuration (config) directory.  One of these is named reliafreecom.sql and the other is example1.sql.  Execute the following to create a common database on the selected MySQL server: \\

    mysql -h <YOUR MYSQL SERVER HOST> -u root -p < reliafreecom.sql

    mysql -h <YOUR MYSQL SERVER HOST> -u root -p < example1.sql

\bigskip \noindent As discussed above, these databases do not need to reside on the same server.  Each of the MySQL database tables are discussed in greater detail in the following sections.

\subsection{\bf \large Model Input Data Tables}

\noindent Model input data is common data.  Therefore, these tables reside in the common database (reliafreecom). \\

    \begin{landscape}
    \begin{tabular}{p{2.0in} | m{1.5in} | m{1.5in} | p{3.75in}}
    \textbf{TABLE} & \textbf{APPLICABLE PREDICTION MODEL} & \textbf{ASSOCIATED ``PI'' FACTOR} & \textbf{NOTES} \\
    \hline
    tbl\_active\_environs & MIL-HDBK-217 & pi\_E & This table shall be indexed such that the selection of a component category, subcategory, and failure rate model return a unique list of environment codes, descriptions, and "pi" values. \\ \hline
    tbl\_application & MIL-HDBK-217 & pi\_A & For GaAs Digital, GaAs MMIC, High Frequency Diode, Low Frequency Bipolar Transistor, Low Frequency Si FET, and High Frequency GaAs FET components. \\
    & MIL-HDBK-217 & pi\_M & For High Power, High Frequency Bipolar Transistor and High Frequency GaAs FET components. \\
    & MIL-HDBK-217 & pi\_U & For Circuit Breaker components. \\
    & MIL-HDBK-217 & lambda\_b & For Low Frequency Diode and Detector/Emitter/Isolator Optoelectronic components. \\
    & MIL-HDBK-217 & K3 & For Mechanical Relays, Toggle Switch, Sensitive Switch, Rotary Switch, and Thumbwheel Switch components. \\ \hline
    tbl\_category & Any & Component (part) category (type) data. & \\ \hline
    tbl\_construction & MIL-HDBK-217 & pi\_C & For \\ \hline
    & MIL-HDBK-217 & pi\_CF & For \\ \hline
    tbl\_cycles & MIL-HDBK-217 & & \\ \hline
    tbl\_dormant\_environs & & & \\ \hline
    tbl\_ecc\_type & MIL-HDBK-217 & pi\_ECC & For EEPROM type memory components. \\ \hline
    tbl\_elements & MIL-HDBK-217 & C1 & For Integrated Circuit category components. \\ \hline
    tbl\_family & MIL-HDBK-217 & Ea & For Integrated Circuit category, Logic subcategory components. \\ \hline
    tbl\_insulation\_class & MIL-HDBK-217 & & \\ \hline
    tbl\_manufacturing\_process & MIL-HDBK-217 & & \\ \hline
    tbl\_package & MIL-HDBK-217 & I5 and I6 & For Integrated Circuit category components. \\ \hline
    & MIL-HDBK-217 & pi\_PT & For Integrated Circuit category, VLSI subcategory components. \\ \hline
    tbl\_quality & MIL-HDBK-217 & pi\_Q & \\ \hline
    tbl\_resistance\_rng & MIL-HDBK-217 & & \\ \hline
    tbl\_specification & MIL-HDBK-217 & & \\ \hline
    tbl\_specsheet & MIL-HDBK-217 & & \\ \hline
    tbl\_subcategories & Any & Component (part) subcategory (sub-type) data. & This table shall be indexed such that a selected category returns only it's associated subcategories. \\ \hline
    tbl\_technology & MIL-HDBK-217 & & \\ \hline
    tbl\_voltage\_ratio & MIL-HDBK-217 & & \\ \hline
    \hline
    \end{tabular}
    \end{landscape}

\subsection{\bf \large Calculation Selection Data Tables}

\noindent Calculation selection data is common data.  Therefore, these tables reside in the common database (reliafreecom). \\

    \begin{landscape}
    \begin{longtable}{ll}
    TABLE & DESCRIPTION \\
    \hline
    tbl\_calculation\_model & This table shall store an indexed list of reliability prediction (calculation) models available to RTK. \\
    tbl\_cost\_type & This table shall store an indexed list of cost types available to RTK. \\
    tbl\_distributions & This table shall store an indexed list of probability distributions available to RTK. \\
    tbl\_hr\_type & This table shall store an indexed list of hazard rate types available to RTK. \\
    tbl\_mttr\_type & This table shall store an indexed list of mean time to repair types available to RTK. \\
    tbl\_part\_classification & This table shall store an indexed list of classification of parts. \\
    \hline
    \end{longtable}
    \end{landscape}

    \begin{landscape}
    \begin{longtable}{ll}
    TABLE & BASE DATA \\
    \hline
    tbl\_calculation\_model & MIL-HDBK-217FN2 Part Stress \\
    & MIL-HDBK-217FN2 Part Count \\
    & NSWC/LE- \\
    tbl\_cost\_type & Calculated \\
    & Specified \\
    tbl\_distributions & Exponential \\
    & LogNormal \\
    & Uniform \\
    & Weibull \\
    tbl\_hr\_type & Allocated \\
    & Calculated, Data \\
    & Estimated, Model \\
    & Similar Item \\
    & Specified, Failure Rate \\
    & Specified, MTBF \\
    tbl\_mttr\_type & Calculated \\
    & Specified \\
    tbl\_part\_classification & Library \\
    & User-Defined \\
    \hline
    \end{longtable}
    \end{landscape}

\subsection{\bf \large System Information Tables}

\noindent System information data is program specific.  Therefore, these tables reside in each of the program databases. \\

    \begin{landscape}
    \begin{longtable}{ll}
    TABLE & DESCRIPTION \\
    \hline
    tbl\_revisions & This table shall store \\
    tbl\_functions & This table shall store \\
    tbl\_functional\_matrix & This table shall store \\
    tbl\_system & This table shall store \\
    tbl\_prediction & This table shall store \\
    tbl\_program\_info & This table shall store \\
    \hline
    \end{longtable}
    \end{landscape}

    \begin{landscape}
    \begin{longtable}{ll}
    TABLE & BASE DATA \\
    \hline
    tbl\_revisions & Dash (original) revision information shall be populated when the database is created. Non-default values shall be as follows: fld\_revision\_code = '-' \\
    tbl\_functions & No information shall be populated when the database is created. \\
    tbl\_functional\_matrix & No information shall be populated when the database is created. \\
    tbl\_system & Top level assembly (system) shall be populated when database is created. Non-default values shall be as follows: fld\_assmbly\_noun = 'System' \\
    tbl\_prediction & No information shall be populated when the database is created. \\
    \hline
    \end{longtable}
    \end{landscape}

\subsection{\bf \large FMECA Information Tables}

\noindent FMECA information data is program specific.  Therefore, these tables reside in each of the program databases. \\

    \begin{landscape}
    \begin{longtable}{ll}
    TABLE & DESCRIPTION \\
    \hline
    tbl\_fmeca & This table shall store \\
    tbl\_fmeca\_causes & This table shall store \\
    tbl\_fmeca\_effects & This table shall store \\
    tbl\_fmeca\_items & This table shall store \\
    tbl\_fmeca\_modes & This table shall store \\
    \hline
    \end{longtable}
    \end{landscape}

    \begin{landscape}
    \begin{longtable}{ll}
    TABLE & BASE DATA \\
    \hline
    tbl\_fmeca & One FMEA of type function shall be populated when the database is created. The initial functional FMEA shall be ID = 1. \\
    & One FMECA of type component shall be populated when the database is created.  The initial component FMECA shall be ID = 2. \\
    tbl\_fmeca\_causes & No information shall be populated when the database is created. \\
    tbl\_fmeca\_effects & No information shall be populated when the database is created. \\
    tbl\_fmeca\_items & No information shall be populated when the database is created. \\
    & A record shall be added to the table for FMEA ID = 1 when a new function is added to tbl\_functions. \\
    & A record shall be added to the table for FMECA ID = 2 when a new part is added to tbl\_prediction. \\
    tbl\_fmeca\_modes & No information shall be populated when the database is created. \\
    \hline
    \end{longtable}
    \end{landscape}

\subsection{\bf \large FRACA Information Tables}

\noindent FRACA information data is program specific.  Therefore, these tables reside in each of the program databases.

\subsection{\bf \large Mission Information Tables}

\noindent Mission information data is program specific.  Therefore, these tables reside in each of the program databases.

\section{\bf \Large RTK Coding Standards}

\noindent General Python coding standards for The RTK Project shall follow the guidelines found in \href{http://www.python.org/dev/peps/pep-0008/ PEP-0008}.  Where PEP-0008 provides options, the following guidance applies.

\begin{enumerate}
    \item Indent with spaces only.
    \item Indent with 4 spaces.
    \item Maximum line length is 79 characters.
    \item Maximum line length for long blocks of text is 72 characters.
    \item Module names use lowercase convention (i.e., modules should have one, short name)
    \item Class names use CapWords convention
    \item Function names use lower\_case\_with\_underscores convention
    \item Global, module level variables use \_leading\_underscore convention
    \item Module level constants use UPPER\_CASE\_WITH\_UNDERSCORE convention
\end{enumerate}

\subsection{\bf \large Component Classes}

\noindent A Python meta-class shall be created for each component category.  A Python class shall be created for each component subcategory.  The category meta-class shall be inherited by each associated subcategory class. \\

\noindent Currently the component category/subcategory class relationships are: \\

    \begin{landscape}
    \begin{longtable}{lcc}
    CATEGORY & SUBCATEGORY & RELIAFREE MODULE FILE \\
    \hline
    Integrated Circuit & & ic.py \\
    & GaAs & gaas.py \\
    & Linear & linear.py \\
    & Logic & logic.py \\
    & Memory & memory.py \\
    & Microprocessor and Microcontrollers & microprocessor.py \\
    & VHLSI and VHSCI & vhsci.py \\
    \hline
    Semiconductor & & semiconductor.py \\
    & Diode & diode.py \\
    \hline
    Resistor & & resistor.py \\
    & Metal Film & film.py \\
    \hline
    Capacitor & & capacitor.py \\
    & Ceramic & ceramic.py \\
    \hline
    \end{longtable}
    \end{landscape}

\noindent

\subsection{\bf \large RTK PyGTK Widget Conventions}

\noindent When naming a PyGTK widget, it will depend on whether the widget is local to the function or is expected to be used module-wide.  Minimize the use of module-wide widgets.  It is preferred to create a high level module-wide widget and use that widget's functions to retrieve lower level widgets.  For example, a treeview may be module-wide and the get\_model() function should be used to retrieve the underlying gtk.TreeModel when it is needed. \\

\noindent For module-wide widgets, a lower-case prefix depicting the type of widget is prepended to a name descriptive of the widget's function.  CapCase convention shall be used for the descriptive name.  Examples of this convention are: \\

    winPartsList
    tvwRevisionTree
    cmbPartCategory

Table XXX, provides PyGTK widget naming conventions for widgets used locally and module-wide. \\

    \begin{landscape}
    \begin{longtable}{lcc}
    WIDGET TYPE & LOCAL NAME & MODULE-WIDE PREFIX \\
    \hline
    gtk.AboutDialog & dialog & dlg \\
    gtk.Alignment & alignment & \\
    gtk.Button & button & btn \\
    gtk.ButtonBox & buttonbox & \\
    gtk.CellRenderer & cell & cel \\
    gtk.CheckButton & checkbutton & chk \\
    gtk.Combo & combo & cmb \\
    gtk.ComboBox & combo & cmb \\
    gtk.ComboBoxEntry & combo & cmb \\
    gtk.Dialog & dialog & dlg \\
    gtk.Entry & entry & txt \\
    gtk.Fixed & fixed & \\
    gtk.Frame & frame & fra \\
    gtk.HBox & hbox & \\
    gtk.HPaned & hpaned & \\
    gtk.IconFactory & iconfactory & \\
    gtk.Label & label & lbl \\
    gtk.Layout & layout & lyo \\
    gtk.ListStore & store & \\
    gtk.Menu & menu & mnu \\
    gtk.MenuItem & menuitem & \\
    gtk.MenuToolButton & menutoolbutton & \\
    gtk.MessageDialog & dialog & dlg \\
    gtk.Notebook & notebook & nbk \\
    gtk.RadioButton & radiobutton & rdo \\
    gtk.ScrolledWindow & scrollwindow & \\
    gtk.SpinButton & spinbutton & spn \\
    gtk.TextView & text & txt \\
    gtk.TreeIter & iter & \\
    gtk.TreeModel & model & \\
    gtk.TreeStore & store & \\
    gtk.TreeView & treeview & tvw \\
    gtk.TreeViewColumn & column & col \\
    gtk.VBox & vbox & \\
    gtk.VPaned & vpane & \\
    gtk.Window & window & win \\
    \hline
    \end{longtable}
    \end{landscape}

\noindent When loading gtk.Combo, gtk.ComboBox, and gtk.ComboBoxEntry, a blank entry shall be included as index 0. \\

\noindent Generic functions for creating, deleting, or interfacing with PyGTK widgets shall be maintained in the file named widgets.py.  This file shall be imported by any other RTK module needing to create, delete, or interface with PyGTK widgets.  These generic functions will connect the created widget's default signal to a callback function.  Other signal/callbacks will be connected in the module calling the generic function. \\

\noindent Example generic function for the gtk.Entry widget: \\

        make\_entry(max=0, height=30, width=200, callback)

            entry = gtk.Entry(max)
            entry.set\_property(height=height)
            entry.set\_property(width=width)
            entry.connect("editing-done", callback)

            return(entry)

\section{\bf \Large RTK GUI Layout}

\noindent The layout of the RTK GUI shall follow the guidance in the subsequent sections.

\subsection{\bf \large Tree Book}

\noindent There are four trees in the Tree Book.  These four trees are referred to as the Revision Tree, the Function Tree, the Hardware Tree and the FRACA Tree.  They are created programatically using a general function in widgets.py.  Which columns are editable, which columns are visible, and the order the columns are displayed is set on a per user basis using the RTK format configuration file (default is format.conf). \\

\noindent The Revision Tree shall be the first tree in the Tree Book.  The Revision Tree shall be used to locally store tbl\_revisions data. \\

\noindent The Function Tree shall be the second tree in the Tree Book.  The Function Tree shall be used to locally store tbl\_functions data. \\

\noindent The Hardware Tree shall be the third tree in the Tree Book.  The Hardware Tree shall be used to locally store tbl\_system data. \\

\noindent The FRACA Tree shall be the fourht tree in the Tree Book.  The FRACA Tree shall be used to locally store tbl\_system data.  It stores the same information as the Hardware Tree, but may be displayed in a different order or with different columns visible than the Hardware Tree. \\

\subsection{\bf \large Parts List}

\noindent The Parts List shall be used to locally store tbl\_prediction data.  Which parts are loaded in the Parts List is determined by the active Tree Book tab.  When the Revision Tree tab is selected, the Parts List contains those parts associated with the selected revision.  When the Function Tree tab is selected, the Parts List contains those parts associated with the selected function.  When the Hardware Tree or FRACA Tree tab is selected, the Parts List contains those parts immediately associated with the selected assembly. \\

\subsection{\bf \large Work Book}

\noindent There are seven tabs in the Work Book.  These tabs are referred to as the General Data tab, the Function Matrix tab, the Calculation Inputs tab, the Calculation Results tab, the FMECA Worksheet tab, the Design Verification tab, and the FRACA Incident tab. \\

\noindent The General Data tab shall be the first tab listed in the Work Book.  The General Data tab shall be divided into quadrants. \\

    \begin{longtable}{cll}
    QUADRANT & LOCATION & DISPLAYS \\
    \hline
    1 & Upper left & General usage information \\
    2 & Upper right & Requirements information \\
    3 & Lower left & Manufacturer information \\
    4 & Lower right & FUTURE USE \\
    \hline
    \end{longtable}

The widgets needed to display information shall be dynamically created when an item in one of the following trees is selected: \\

    - Revision Tree
    - Function Tree
    - Hardware Tree
    - FRACA Tree
    - Parts List

\noindent The Function Matrix tab shall be the second tab listed in the Work Book.  The widgets needed to display information shall be dynamically created when the Function Tree is selected. \\

\noindent The Calculation Inputs tab shall be the third tab listed in the Work Book.  The Calculation Inputs tab shall be divided into quadrants. \\

    \begin{longtable}{cll}
    QUADRANT & LOCATION & DISPLAYS \\
    \hline
    1 & Upper left & Reliability model inputs \\
    2 & Upper right & Burn-in inputs \\
    3 & Lower left & Junction temperature inputs \\
    4 & Lower right & Derating inputs \\
    \hline
    \end{longtable}

The widgets needed to display information shall be dynamically created when an item in one of the following trees is selected: \\

    - Hardware Tree
    - Parts List

\noindent The Calculation Results tab shall be the fourth tab listed in the Work Book.  The Calculation Results tab shall be divided into quadrants. \\

    \begin{longtable}{cll}
    QUADRANT & LOCATION & DISPLAYS \\
    \hline
    1 & Upper left & Reliability calculation results \\
    2 & Upper right & Maintainability calculation results \\
    3 & Lower left & Combined with quadrant 1 to create rectangular region for reliability results \\
    4 & Lower right & Miscellaneous calculation results (e.g., cost) \\
    \hline
    \end{longtable}

The widgets needed to display information shall be dynamically created when an item in one of the following trees is selected: \\

    - Revision Tree
    - Function Tree
    - Hardware Tree
    - FRACA Tree
    - Parts List

\noindent The FMECA Worksheet tab shall be the fifth tab listed in the Work Book.  The widgets needed to display information shall be dynamically created when an item in one of the following trees is selected: \\

    - Function Tree (functional FMEA)
    - Hardware Tree (component FMECA)

\noindent The Design Verification tab shall be the sixth tab listed in the Work Book. \\

\noindent The FRACA Incident tab shall be the seventh tab listed in the Work Book. \\

\section{\bf \Large RTK Calculations}

\subsection{\bf \large Reliability Predictions}

\subsubsection{\bf MIL-HDBK-217 Part Stress}
\subsubsection{\bf MIL-HDBK-217 Part Count}
\subsubsection{\bf NSWC/LE-}
\subsubsection{\bf Dormant}
\subsubsection{\bf Similar Item}

\subsection{\bf \large Criticality}
\subsubsection{\bf RPN Criticality}
\subsubsection{\bf MIL-STD-1629A, Task 102 Criticality}

\subsection{\bf \large Risk}
\subsubsection{\bf RPN-Based Risk}
\subsubsection{\bf MIL-STD-882D Risk}

\subsection{\bf \large Life-Data}

\noindent FUTURE

\subsection{\bf \large Fault Tree}

\noindent FUTURE

\end{document}
